\documentclass[FM,DP]{tulthesis}

\usepackage[utf8]{inputenc}
\usepackage[czech]{babel}
\usepackage{setspace}
\onehalfspacing

\TULtitle{Editor konfiguračních souborů Flow123d}{Editor for Flow123d configuration files}
\TULprogramme{N2612}{Elektrotechnika a informatika}%
{Electrotechnology and informatics}
\TULbranch{1802T007}{Informační technologie}%
{Information technology}
\TULauthor{Bc. Tomáš Křížek}
\TULsupervisor{doc. Ing. Jiřina Královcová, Ph.D.}  %TODO uvadet vsechny tituly?
\TULyear{2016}

\begin{document}
\ThesisStart{male}

\begin{abstractCZ}
\thispagestyle{empty}
Český abstrakt
\end{abstractCZ}

\vspace{2cm}
\begin{abstractEN}
English abstract
\end{abstractEN}

\clearpage
\begin{acknowledgement}

\end{acknowledgement}

\tableofcontents
\clearpage

\begin{abbrList}
\textbf{EU} & Evropská unie \\
\end{abbrList}

\chapter*{Úvod}

%TODO - formulace "mel by"? nebude lepsi to psat jinak?

Existuje celá řada softwarů, která pro zajištění požadované funkce potřebuje správné nastavení, podle kterého pak daný program přizpůsobí svoji činnost. Může se jednat o~počáteční konfiguraci, jako je tomu například u~serverových aplikací nebo e-mailových klientů. Tato konfigurace se zpravidla dále nemění, pokud nedojde k nějakým podstatným změnám.

Oproti tomu existují programy, od kterých se očekává, že budou spouštěny s širokou škálou různých nastavení. U~těchto aplikací se typicky konfigurace předává při spuštění jako jeden ze vstupních parametrů. Činnost těchto programů se pak zásadně liší dle zvolené konfigurace.

%TODO jak popsat flow?
%TODO jake jsou mozne typy uloh?
Takovou aplikací je například simulátor Flow123d, který pracuje se zadanou výpočetní sítí, na které provádí výpočty dle definované úlohy. Úloha se definuje pomocí konfiguračního souboru, který pak spolu s~výpočetní sítí a ostatními daty tvoří vstupní data pro aplikaci Flow123d.

Vzhledem k tomu, že Flow123d podporuje různé typy úloh a obsahuje rozsáhlé dílčí nastavení, vyvstává otázka toho, zda je zvolená konfigurace validní. Uživatel může omylem zvolit kombinace hodnot, které nejsou přípustné, ať už taková chyba vznikne logicky nebo například kopírováním. Dalším běžným případem chyby je překlep, kterého si uživatel nemusí všimnout.

%TODO logicka vs semanticka chyba?

Formát konfiguračních souborů pro Flow123d je poměrně rozsáhlý -- samotná referenční příručka, která ho popisuje, obsahuje několik desítek stran. Čím je složitější formát pro konfiguraci, tím je pro člověka náročnější ověřit, zda byly zadány všechny povinné parametry, nebo zda při upravování souboru nedošlo k nějaké nepatrné změně, která do souboru zanesla chybu. Vzhledem ke složitosti formátu jsou na uživatele kladeny velké nároky -- buď musí mít se softwarem rozsáhlé zkušenosti, nebo musí trávit čas prohledáváním dokumentace.

%TODO je to vhodna formulace?
Celá situace je dále komplikována tím, že formát konfiguračních souborů se mění s~tím, jak se vyvíjí funkce softwaru Flow123d. Některé změny ani nejsou zpětně kompatibilní. Tím pádem soubor, který byl validní pro starší verzi už nemusí být validní pro nově vydanou verzi. Uživatele tedy opět čeká studium rozsáhlé referenční dokumentace, aby zjistil, jaké změny má provést.

Pokud se stane, že uživatel spustí Flow123d s nevalidní konfigurací, tak během inicializaci dojde k chybě, o které se uživatel dozví pomocí textového rozhraní, ve kterém se Flow123d spouští. Jelikož se může jednat o~výpočetně náročné úlohy, které se často pouští na vzdáleném výpočetním clusteru, je tento proces poměrně zdlouhavý. Pokud se úloha spouští vzdáleně, musí dojít k~navázání komunikace a přidělení zdrojů, než může vůbec dojít k inicializaci úlohy, při které pak může dojít k chybě. Odhalování takových chyb je pak časově náročné a uživatelsky nepříjemné.

Tyto důvody byly hlavní motivací ke vzniku speciálního editoru pro konfigurační soubory Flow123d, který práci s nimi značně zjednoduší a usnadní. Aplikace by měla zrychlit proces odstranění chyb v~konfiguračních souborech tím, že je umožní odhalit už v průběhu jejich vytváření nebo upravování. To dává uživateli možnost chyby opravit ještě před tím, než předloží konfigurační soubor softwaru Flow123d. To vede ke značné časové úspoře obzvlášť v případech, kdy se výpočetní úloha spouští vzdáleně.

Editor by měl uživateli poskytovat přívětivé uživatelské rozhraní, které mu zjednoduší přístup k~dokumentaci. V rámci aplikace by měl mít uživatel k dispozici dokumentaci, která bezprostředně souvisí s právě upravovanou částí konfiguračního souboru. Tato forma nápovědy by měla uživateli poskytnou alternativu k manuálnímu prohledávání několika stránkové referenční dokumentace.

Dále by editor měl umožnit vizualizaci datové struktury, která tvoří konfigurační soubor. Kromě toho se od editoru očekávají základní funkce pro práci s~textovými soubory, jako je podpora operací se schránkou, možnost vrátit či opakovat změny, vyhledávání či nahrazení textu a další. Editor by měl podporovat platformy Windows a Linux.

%TODO zminit strukturu prace - rozcestnik?


\chapter{Rešerše}

%TODO nastaveni flow - serializace objektu - binarni vs textovy format

%TODO nekde zminit - jednosmerna konverze, protoze neplati WYSIWYG, nepracuje se s DOM ale primo s textem



\section{Datové struktury pro popis dat}

%TODO uvod do flow na urovni znalosti potrebne pro moji praci
% nastinit autokonverze, abstraktni zaznamy (zminit ze nemaji dedicnost, spise interface)
% serializace dat -> zaznamy, pole, hodnoty ... obrazek stromu?
% vztah a vznik souboru s formatem - zavislost na verzi

%TODO jak to nazvat, aby nikdo neocekaval, ze tady zacnu rozebirat binarni stromy? 

\section{Jazyky pro popis dat}
%TODO moznosti pro serializaci objektu a repr. konf. souboru
% binarni vs textovy format
% XML, DTD, XMLSchema
% JSON, JSON Schema
% YAML
% proc byl zvolen YAML, diskuze ostatnich moznosti

\section{Formát konfiguračních souborů}
% o co se jedna, jak to vypada (JSON)
% jakym zpusobem popisuje data - hiearchicka stromova struktura
% datove typy - scalar, array, record, abstract
% autokonverze + priklady?


\chapter{Analýza}


\section{Konfigurační soubory}
Priblizeni konfiguracnich souboru a jejich formatu.

Stary format CON a prechod na novy format YAML.

Diskuze XML/YAML(JSON). 

\section{Input Structure Tree}
Popis formatu datovych souboru.

Definice konkretnich datovych typu.

Specifikovat mozne atributy.

\section{Autokonverze}
Popsat mozne konverze, uvest do souvislosti s~XML transformacemi?


\chapter{Návrh}
Diagramy pro zpracovani YAML (inspirace SAX vs DOM).




\chapter{Implementace}


\chapter*{Závěr}
V rámci této diplomové práce byl vytvořen editor konfiguračních souborů pro Flow123d. Jedná se o samostatně funkční aplikaci, která je ovšem navržena s ohledem na její použití jako součást softwarového balíku GeoMop, který obsahuje další nástroje, které usnadňují práci uživatelům Flow123d.
%TODO zminovat GeoMop?

Editor uživatelům zjednodušuje vytváření a upravování konfiguračních souborů. Umožňuje ověřit správnost zadané konfigurace pro zvolenou verzi Flow123d a případně uživatele upozornit na detekované chyby. Tato funkce uživateli přináší časovou úsporu a uživatelsky příjemnější rozhraní při odhalování chyb.

Editor dále uživatelům poskytuje kontextovou dokumentaci a našeptávač. Obě tyto funkce přizpůsobují svůj obsah na základě pozice kurzoru v textu, tedy oblasti, kterou uživatel zrovna upravuje. Pro uživatele to představuje značné zjednodušení, jelikož může využít tyto funkce místo prohledávání rozsáhlé dokumentace.

V neposlední řadě editor obsahuje komponentu pro grafické znázornění datové struktury, která poskytuje alternativní pohled na zadaná data, a umožňuje rychlejší orientaci v rozsáhlých konfiguračních souborech. Kromě těchto stěžejních funkcí editor poskytuje i běžné nástroje pro manipulaci s textem, jako jsou například operace se schránkou, možnost vracení provedených změn, vyhledávání a nahrazení nebo změna úrovně odsazení.

%TODO zmenit formulaci
Aplikace je multiplatformní a podporuje systémy Windows (XP nebo novější) a Linux. S ohledem na požadavek multiplatformní aplikace byl pro vývoj použit jazyk Python~3 a grafická knihovna PyQt~5. K aplikaci byly vytvořeny instalační balíčky pro Windows a Debian.

V rámci budoucího vývoje jsou plánovány další dodatečné funkce. Jedná se např. o zlepšení zvýraznění syntaxe, které se by se mohlo přizpůsobit přímo formátu Flow123d. Další možné vylepšení spočívá v rozšíření funkcionality komponenty pro vizualizaci datové struktury. Ta by mohla v budoucnu podporovat kromě zobrazení i editaci dat nebo vylepšené zobrazení speciálních datových typů.



\end{document}
